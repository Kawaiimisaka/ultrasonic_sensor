	%------------------第一章---------------------------
	\newpage
	\section{绪\texorpdfstring{\quad}{} 论}
    \subsection{研究背景}
    \songti\zihao{-4}{
    随着社会不断发展,劳动力成本不断升高,人们对自动化生产需求也随之变高。而在自动化生产的过程中,使物料能够准时准确到达指定位置就显得极为关键,这直接关系着生产安全和生产效率。\par
    超声波接近传感器被用于检测不同材料、外形、颜色或浓度的物体,具有极佳的精确性、灵活性和可靠性,被广泛应用于无人机、自动化生产线等领域。相比其他检测方法,超声波检测可以对各种不同特性的物体进行检测,比如固体、液体、粉末,甚至是透明物体。检测与表面的性质无关,表面可以粗糙或平滑、清洁或脏污、潮湿或干燥。而且其结构非常坚固,对脏物、环境光线或噪声不敏感。本毕业设计拟基于TUSS4470集成芯片开发设计超声波接近传感器,主要用于钢化玻璃自动化生产线到位检测。
    
    }
    \subsection{研究目的与意义}
    \songti\zihao{-4}{
    超声波传感器的检测精度直接取决于其发出脉冲宽度的精度\upcite{1},因此控制产生精确的脉冲宽度对提高检测精度有着极为重要的作用。\par
    在传统的超声波驱动控制电路中,一般是采用模拟电路或者单片机来控制。由模拟电路驱动的超声波传感器抗干扰性差,而由单片机驱动的超声波传感器,由于其使用外部中断触发的机制,导致无法精确控制时序逻辑,从而难以达到与超声探头匹配的驱动频率,最终使得传感器的精度降低。\par
    本设计采用型号为EPM240T100C5N的MAX II系列芯片,它是一种高集成度、电可擦除、CMOS宏阵列可编程逻辑器件\upcite{CPLD芯片介绍},可以产生ns级别的控制信号,配合TUSS4470超声驱动芯片,可精确控制发送脉冲的次数、频率以及脉冲宽度。同时,CPLD芯片编程采用时序逻辑,发送、接收、检测脉冲信号的时间可进行精确控制,这让超声检测策略可以更加丰富合理。
    }
    \subsection{研究思路与方法}