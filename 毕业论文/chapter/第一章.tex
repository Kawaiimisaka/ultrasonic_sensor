	%------------------第一章---------------------------
	\newpage
	\section{绪论}
    \subsection{超声波接近传感器研究背景}
    \subsubsection{超声波接近传感器发展及原理}

超声波是一种机械波,是在频率高于人类听力范围(20kHz)的范围内发生的声波。超声波技术的发现和应用,主要是由于第二次世界大战期间雷达技术的发展而兴起的。\par
超声波传感器是利用超声波的传播特性进行测量的一种传感器。其最早的应用是在医学领域,用于检测人体内部器官的情况。后来,随着科技的发展,超声波传感器得到了广泛的应用,包括汽车工业、制造业、环保等领域。传统超声波传感器的局限性:传统超声波传感器存在着以下局限性:易受到环境干扰、测量范围受到限制、精度有待提高等。为了解决传统超声波传感器的局限性,研究人员提出了一系列新型的超声波传感器,例如基于MEMS技术的微型超声波传感器、基于谐振腔的超声波传感器、基于光纤传输的超声波传感器等。

在现代工业和自动化控制中,超声波传感器已经成为了一种重要的非接触式测量和检测手段。传统的超声波传感器由于受到环境干扰和传感器本身精度等问题的限制,其应用场景和测量范围受到一定的限制。因此,需要一种高精度、低功耗、多功能的超声波传感器来满足实际应用需求。\par
TUSS4470芯片作为一种新型的超声波传感器芯片,具有高精度、低功耗和多种工作模式等特点,可以满足现代工业和自动化控制对于测量、检测、控制和导航等方面的需求。因此,基于TUSS4470芯片的超声波接近传感器的研究成为了一个热门话题,其可以应用于智能家居、无人机、自动驾驶车辆、机器人等领域,具有广阔的应用前景和市场前景。\par
此外,随着智能制造和智能物流等新兴产业的发展,人们对于传感器的要求也越来越高。因此,开发一种基于TUSS4470芯片的超声波接近传感器,对于推动智能制造和智能物流等领域的发展,提高生产效率和产品质量有着重要的意义。
   
    \subsubsection{超声波接近传感器国内外研究现状}
    \subsection{超声波接近传感器研究目的与意义}

    超声波传感器的检测精度直接取决于其发出脉冲宽度的精度\upcite{1},因此控制产生精确的脉冲宽度对提高检测精度有着极为重要的作用。\par
    在传统的超声波驱动控制电路中,一般是采用模拟电路或者单片机来控制。由模拟电路驱动的超声波传感器抗干扰性差,而由单片机驱动的超声波传感器,由于其使用外部中断触发的机制,导致无法精确控制时序逻辑,从而难以达到与超声探头匹配的驱动频率,最终使得传感器的精度降低。\par
    本设计采用型号为EPM240T100C5N的MAX II系列芯片,它是一种高集成度、电可擦除、CMOS宏阵列可编程逻辑器件\upcite{2},可以产生ns级别的控制信号,配合TUSS4470超声驱动芯片,可精确控制发送脉冲的次数、频率以及脉冲宽度。同时,CPLD芯片编程采用时序逻辑,发送、接收、检测脉冲信号的时间可进行精确控制,这让超声检测策略可以变得更加丰富合理。
    
    \subsection{超声波接近传感器研究思路与方法}
    本设计的研究思路和方法按照如下所示:\par
   1、确定系统设计需求:首先需要明确设计需求,包括测量范围、测量精度、输出格式等要求,以及硬件和软件的设计要求等。

2、确定硬件平台:根据设计需求,选择合适的硬件平台。可以选择使用CPLD芯片来控制TUSS4470芯片,实现测距和数据处理等功能。同时,还需要选择合适的电源、滤波电路、放大电路等外围电路,以满足系统的实际应用需求。

3、确定软件设计:根据硬件设计要求,编写相应的软件程序,实现数据采集、处理、存储和输出等功能。在软件设计过程中需要考虑到系统的实时性、准确性和稳定性等要求。

4、搭建系统原型:基于上述硬件和软件设计,搭建系统原型进行实验测试和验证。根据实验结果调整系统设计,优化算法和参数等。

5、系统集成和优化:根据实际应用需求,对系统进行进一步的集成和优化,包括系统的可靠性、稳定性、抗干扰性等方面的优化。

6、系统测试和验证:对系统进行全面的测试和验证,包括系统的性能测试、功能测试、可靠性测试等,以确保系统的稳定性和可靠性。\par

总之,制作基于TUSS4470芯片的超声波接近传感器的研究思路和方法需要综合考虑硬件和软件设计的方方面面,并结合实验测试和验证来不断优化和完善系统设计。