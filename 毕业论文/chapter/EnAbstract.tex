%!TEX root = ../main.tex
\begin{EnAbstract}
    
    This design aims to solve the problem of detecting the positioning of tempered glass on the production line. A highly stable, flexible and widely applicable ultrasonic proximity sensor is developed. Compared with traditional single-chip control, this design uses a CPLD chip to generate sensor driving control signals. Combined with the TUSS4470 integrated chip, higher precision and more stable driving signals can be obtained. At the same time, the design uses a detection strategy of multiple pulse waves and a set detection threshold to judge the detection status. By adjusting the number of pulse wave emissions and the number of pulses, the flexibility and reliability of the sensor in production applications are improved.
    
    The design aims to create an ultrasonic proximity sensor capable of detecting objects within the range of $100mm \sim 200mm$ with a detection accuracy of $1mm$.
    
    This article will detail the design process and working principle of the ultrasonic proximity sensor's hardware and software components. After completing the simulation of the hardware and software components, physical welding and production will be performed, and the program will be debugged using an oscilloscope. Finally, an experimental plan will be designed to test the sensor's performance parameters and echo characteristics.
  
  
    \EnglishKeyWord ultrasonic proximity sensor, CPLD chip, TUSS4470 integrated chip , detection strategy
\end{EnAbstract}