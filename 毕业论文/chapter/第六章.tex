\newpage
\section{总结与展望}
在本研究中,我们成功地基于TUSS4470芯片设计并制作了超声波接近传感器,并使用CPLD芯片实现了传感器的控制电路。我们详细地介绍了超声波传感器的工作原理和设计流程,并对CPLD芯片的控制电路、电源电路、JTAG下载模块和时钟模块等进行了详细的分析和讨论。通过实验验证,我们证明了该传感器具有高精度、高灵敏度、低功耗等优点,可应用于机器人导航、工业自动化、智能家居等领域。\par
尽管本设计完成了一定成果,但仍然存在一些局限性和不足之处。未来的研究可以从以下几个方面进行深入研究:\par
将CPLD芯片用价格低廉、封装小的国产FPGA芯片替代。随着国内FPGA技术的发展和成熟,其在性能、功耗和成本等方面的优势逐渐显现,通过使用国产FPGA芯片,可以进一步降低传感器成本,提高传感器的经济性。同时,在本设计中CPLD芯片的引脚并未被完全利用,存在浪费,可选择引脚更少的国产FPGA,提高引脚的利用率。此外,国产FPGA的封装只有CPLD芯片的$\frac{1}{4}$,可以很大程度上减小超声传感器的尺寸,利于其在生产线上的布置。\par
改进硬件电路设计,本设计在进行实验的过程中,发现PCB电路板存在工作不稳定的情况,经过排查,发现降压芯片的选取不合理,功率不满足电路板的需求,导致电路板容易发热。后续需要优化电路板设计,优化器件选型,提高电路板的稳定性和可靠性。\par
改善PCB布局设计,减小信号间的相互干扰。在本设计的PCB布局中虽然已经考虑到了信号间的隔离,但是仍然存在干扰,后续需要继续优化PCB的设计,可将二层板替换成四层板,从而可以更好的将模拟信号、数字信号、电源信号布局分离,提高信号的稳定性。\par
改进超声波传感器的信号处理算法以及检测策略,进一步提高传感器的精度和灵敏度。在本设计中,一个检测周期进行了五次的检测,耗时$10ms$,而检测策略的选择则需要根据应用场景进行判断,这需要进行大量的实验才可以得出其规律,这都是未来可以进行的研究工作。\par
针对不同领域的应用场景,对传感器进行测试实验。在本设计中,只对于几种材料、距离进行测试实验,然而在实际的应用中,这几组实验是远远不够的,需要对更多类型的材料、距离进行实验测试。如若有需求,还应该搭建更加稳定的实验平台,对于测试物体的检测距离、移动速度都加以更精确的控制。\par


