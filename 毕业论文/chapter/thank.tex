%!TEX root = ../main.tex
\makesection{致谢}
在广东工业大学本科四年的生活即将结束,回首过去四年,没有太多的风花雪月,有的是一个个踏实学习的日日夜夜。衷心感谢在这四年中给我提供帮助的人们。\par
首先我要感谢在大一大二指导过我画法几何的莫春柳老师,老师严谨的教学态度、幽默风趣的教学风格使我印象深刻。在老师指导我的过程中,老师不仅提供了教学上的帮助,还激发了我对本专业知识的浓厚兴趣,为我之后的学习生活提供了动力。\par
感谢机电工程学院仿生与智能机器人实验室的张涛老师,张老师专心科研、认真指导学生,在我大三参与实验室课题期间,每周都会组织组会进行汇报和指导工作。而更让我受益匪浅的是老师对待科研工作的巨大热情,感染着身边的人们,为我心中埋下了科研梦想的种子。\par
感谢我的毕业论文指导老师刘桂贤老师,老师对我们的毕业课题十分认真负责,在老师的指导和督促下,我们按时完成了要求的任务。\par
感谢在人生节点上给我提供建议和帮助的师兄师姐们,他们在我四年的生活中给予了我许多宝贵的意见。\par
感谢我的本科宿舍的室友们,四年里我们互相勉励,互相指导学习,完成了很多科目的课程设计、参加一次又一次科技学术比赛。来到毕业期每个人又各自找到了新的生活目标,去往五湖四海,有望江湖再见。
感谢我的家人对我的支持,在四年里不断对我进行支持和鼓励,为我的求学生涯提供了坚实的物质基础。\par
感谢机电工程学院足球队的队员们,在我心情烦闷时都会陪伴着我,为我提供建议、排忧解难。\par
感谢母校的培养,广东工业大学为我提供了学习的平台,无论今后身在何处,我都会秉持与发扬广工精神,以团结、勤奋、求是、创新的校训作为自己的行事准则,展现广工人“好用、实用、耐用、听话、出活”的人才气质。\par
最后,感谢本论文的评审老师以及参加答辩的专家、老师们,感谢你们对本论文提出的宝贵意见。