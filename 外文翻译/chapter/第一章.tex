\section{外文文献译文}
\subsection{介绍}
随着时间的推移,车辆、移动机器人和生产线逐渐采用了各种传感器。尤其是测量附近物体距离的传感器对智能系统至关重要。目前,激光传感器、超声波传感器、立体摄像头、雷达和光学测距系统广泛应用作为距离传感器。时间飞行(TOF)摄像头也被研究用于提供高帧率的3D成像[1]。其中,超声波传感器非常便宜,传感器系统体积小且易于操作,但由于超声波的物理限制,性能、精度和测量速率通常低于其他传感器系统。因此,软件研究的重点是克服这些物理限制。

大多数超声波距离传感器的研究都集中在通过向超声波中添加附加信息来提高超声波传感器性能。有许多将信息存储在超声波信号中的方法:例如,频率调制、脉冲宽度调制信号编码、脉冲位置调制(PPM)和相位调制。

自上世纪80年代以来,已经进行了许多防止碰撞的研究。在避障领域,Borenstein和Koren [2]–[5]进行了开创性的研究。此后,在多用户问题方面,为提高超声波传感器的精度做出了努力。研究了除简单阈值法以外的各种信号处理方法,用于精确的超声波脉冲检测。几位研究人员使用Barker码、Golay码和Kasami码进行超声波局部定位系统[6]–[10]。交叉相关法用于脉冲检测[11],[12]。Jörg和Berg [13],[15],[16]以及Jörg等人 [14]使用伪随机码来解决串扰问题。然而,这些用于准确的超声波脉冲检测的方法对于汽车应用来说并不足够,因为它们需要高性能微控制器的高计算负荷。

一些研究人员引入了二进制频率移位键控调制以产生发射信号[17]–[20]。频率调制技术需要使用通常价格较高的宽频带压电换能器。

目前,信号编码方法的研究旨在解决移动机器人多超声波传感器系统中的串扰问题。在车辆应用中,当对向车辆在狭窄的单车道街道上使用超声波传感器,或者当城市道路上的车辆侧面有超声波传感器时,可能会发生外部串扰。串扰是超声波传感器实际应用中的一个重要问题。一篇综述性文章[21]介绍了截至2005年的串扰消除方法的研究。区分每个超声波传感器以拒绝串扰是至关重要的。当使用混沌PPM(CPPM)生成一系列超声波脉冲序列时,每个超声波换能器可以变得独特,因此可以区分每个超声波传感器,这意味着消除串扰。

Borenstein和Koren [22],[23]提出了比较连续读数和比较交替延迟的错误消除快速超声波发射的思想。然后,随着计算机处理能力的增加,采用了交叉相关方法来比较序列。Fortuna等人 [24],[25]将CPPM应用于声纳传感器发射,Yao等人 [26]使用混沌脉冲位置宽度调制获得更好的结果。为增强实时性能,Meng等人 [27]提出了短优化的PPM序列。Alonge等人 [28]将雷达技术应用于超声波传感器系统。

与上述研究不同,本文的重点是在多用户环境中提高超声波传感器系统的测量速率。CPPM信号克服了串扰和多用户问题,并通过使用信号编码方法在高测量速率下可稳健地测量物体距离,这与传统的简单TOF方法(单脉冲回波法)不同。此外,车辆应用还需要克服一些障碍。传感器系统必须对环境噪声具有鲁棒性,并能够测量移动目标。基本上,它应在多用户环境中可用,并且最重要的是具有成本竞争力。此外,一般的汽车控制应用要求均方根(rms)误差小于50毫米,并且测量速率超过100赫兹。

在传统的单脉冲回波方法中,当上一个脉冲仍在飞行中时,发射器无法发射附加的超声波脉冲,因为无法区分每个超声波脉冲。本文中,通过比较超声波脉冲序列而不是比较单个超声波脉冲,来获取TOF。在计算TOF时,通过传输脉冲序列和接收到的脉冲信号之间的交叉相关计算时间延迟。然而,通常交叉相关需要高计算成本的长信号。因此,为了降低计算负载,本文采用Hirata等人 [29],[30]引入的单比特信号处理方法在交叉相关过程中降低计算负载并提高实时性能。通过在交叉相关中用单比特逻辑乘积替换多比特乘积,大大减少了计算量。在[29]和[30]中,应用了单比特信号处理来降低检测单个超声波脉冲的计算成本。然而,在本文中,将单比特信号处理应用于CPPM信号序列的交叉相关。

第II节介绍了实验中使用的硬件系统,并解释了经典方法的物理限制。第III节介绍了实验中使用的硬件系统和混沌系统。第IV节介绍了用于实时距离测量的信号处理;噪声抵消采用了快速傅里叶变换(FFT)阈值,子采样和单比特信号处理用于降低计算负载。第V节展示了串扰抑制模拟的结果,并在第VI节给出了测量速率的距离测量实验结果。

\subsection{测量方法}
超声波距离传感器使用TOF(Time of Flight)方法。在最简单的方式中,控制器测量从发射到回波到达的时间。更先进的方法是通过比较发送信号和接收信号来测量信号延迟。信号延迟可以通过交叉相关进行测量,该方法将在本文后面进行讨论。

\begin{equation*} d = C \cdot t_{\text {tof}}/2 \tag{1}\end{equation*}

其中d为目标距离,C为空气中的声速,ttof表示通过该系统测量的声脉冲飞行时间。声速随温度和湿度而变化。线性模型如下:

\begin{equation*} C = 331.3 + 0.606 \cdot \theta + 0.0124 \cdot H \tag{2}\end{equation*}

其中θ为摄氏度温度,H为相对湿度。温度变化会导致声速发生较大变化,因此实时温度测量是超声波距离传感中基本且必要的。然而,湿度偏差被忽略,因为预计其对本文所需的精度影响非常小,不超过0.2%。

\subsubsection{单脉冲回波法}
超声波距离传感器的传统方法称为单脉冲回波法。在超声波发射器发射声脉冲后,控制器等待回波脉冲以测量飞行时间ttof。由于等待回波需要较长时间,测量速率与距离成反比关系。此外,如果回波由于噪声或环境原因消失,将会出现更严重的问题。传感器必须等待超时才能发射下一个声脉冲。

传统单脉冲回波法的最大测量速率如下:
\begin{equation*} \text {最大测量速率} = \frac {1}{t_{\text {tof}}} = \frac {C}{2 \cdot d}. \tag{3}\end{equation*}

由于最大测量速率与距离成反比,对于10米的目标而言,仅为17赫兹。单脉冲方法测量速率受限的主要原因是它仅使用一个声脉冲进行测量。如果在先前的声脉冲飞行过程中超声波传感器发送额外的声脉冲,将在单脉冲回波法中计算错误的TOF。为了包含更多信息,需要通过调制超声信号或超声脉冲序列,而不是一次发送相同的超声波脉冲。在之前的研究[31]中,每个声脉冲通过与上一个声脉冲之间的间隔进行特征化。然而,该方法的缺点是对噪声不够稳健。可以通过对发送和接收信号进行交叉相关来计算TOF,而不是使用单个超声脉冲测量往返时间,从而解决这个问题。

\subsection{测量系统}
图1显示了所提出系统的硬件结构图。发射器和接收器均使用40kHz超声波换能器。微控制器用于运行超声波发射器操作和数据采集。微控制器包括一个1MHz的模拟数字转换器(ADC),对40kHz的超声波进行采样,并带有一个1.33GHz的双核CPU和可编程门阵列模块。微控制器计算Chua电路的微分方程以生成CPPM的混沌信号,并向超声波发射器提供数字脉冲序列。超声波脉冲按预定的混沌脉冲序列触发,收集的超声波数据传输到计算机,并在计算机上进行进一步的信号处理。模拟信号滤波器是一个简单的高通滤波器,用于去除直流分量。

为了消除串扰,采用了混沌系统。CPPM方法被采用将混沌信息并入数字脉冲序列以进行超声波脉冲输出。

\subsubsection{Chua电路}
Chua电路[32]是一个众所周知且最简单的自主电子电路,用于产生混沌信号。Chua电路如图2(a)所示,产生双滚动吸引子[33]。它由电阻、电感、电容和称为Chua二极管的非线性电阻组成,其静态特性如图2(b)所示。图2的状态方程为:

\begin{align*}
	C_{1} \frac {dv_{1}}{dt}=&G (v_{2} - v_{1}) - f(v_{1}) \\
	C_{2} \frac {dv_{2}}{dt}=&G (v_{1} - v_{2}) + i_{L} \\
	L \frac {di_{L}}{dt}=&-v_{2} \tag{4}
\end{align*}

其中,G=1/R。Chua二极管表现出一个只有一个变量的非线性函数,这与具有两个非线性的Lorenz方程不同,每个非线性都是两个变量的标量函数。Chua二极管的非线性函数是一个由三条直线段组成的奇对称分段线性函数:
\begin{equation*}
	f(v_{1})=m_{0}v_{1}+\frac {1}{2}(m_{1}-m_{0})(|v_{1}+B_{p}|-|v_{1}-B_{p}|).\qquad \tag{5}
\end{equation*}
应用以下无量纲变量:
\begin{align*}
	x=&\frac {v_{1}}{B_{p}} \quad y=\frac {v_{2}}{B_{p}} ~ z=\frac {i_{L}}{B_{p}G} \\
	a=&\frac {m_{1}}{G} \quad b=\frac {m_{0}}{G} \\
	\alpha=&\frac {C_{2}}{C_{1}} \quad \beta =\frac {C_{2}}{LG^{2}}\tag{6}
\end{align*}
当时间缩放为τ=tG/C2时,得到无量纲形式的方程组:
\begin{align*}
	\frac {x}{\tau }=&\alpha (y-x-f(x)) \\
	\frac {y}{\tau }=&x-y-z \\
	\frac {z}{\tau }=&-\beta y \tag{7}
\end{align*}
其中,
\begin{equation*}
	f(x)=bx+\frac {1}{2}(a-b)(|x+1|-|x-1|). \tag{8}
\end{equation*}
所使用的变量、初始值和时间缩放因子如下:
\begin{align*}
	\alpha=&9, \quad \beta =\frac {100}{7} ~a=-\frac {8}{7}~b=-\frac {5}{7}\\
	x(0)=&0.09 \quad y(0)=0 ~x(0)=0 ~\frac {G}{C_{2}}=1000.
\end{align*}
(7)式的状态变量x的图形如图3所示。

\subsubsection{混沌脉冲位置调制}
为了创建具有混沌间距的数字信号,混沌模拟信号经过一种称为脉冲位置调制(PPM)的过程,即CPPM。调制机制如图4所示。这种方法在Fortuna等人的超声波距离传感器系统中引入。连续的脉冲位置调制器由采样保持电路和斜坡发生器组成。混沌信号是PPM电路的输入,并进行采样和保持。当保持值减去递增的斜坡信号等于零时,电路生成一个数字脉冲,重置斜坡信号,并再次进行采样和保持。模拟电压的混沌信息转换为数字脉冲之间的间隔。

\subsubsection{信号处理}
图5显示了所提出的测量方法的信号处理过程。图中的发射器部分包括Chua电路和PPM调制器。接收器部分包括超声波接收器、用于噪声过滤的正弦波相关器和用于二值化的阈值。发射器根据CPPM序列生成超声波脉冲。当从发射器发射的超声波反射到目标并被接收器收集时,会发生传输延迟。传输延迟对应于TOF(飞行时间),可以通过两个信号的互相关测量来测量。每个发射器和接收器部分向处理单元提供单比特信号以计算距离。在本文中,接收器部分被配置为采样系统以进行进一步研究,但可以根据需要配置为模拟电路。

FFT用于确定超声波接收器收集到的电压信号中是否存在超声波信号。本文中使用的超声波频率为40 kHz(准确为39.4 kHz)。由于在接收到的电压信号中只需要40 kHz的频域,如果仅计算40 kHz分量,则计算量非常小,可以快速计算。本文中,超声波信号的ADC速度为1 MHz,40 kHz超声波的周期为25 μs。因此,每1 μs获得一个新的电压值,然后,对一个25 μs长的电压信号的40 kHz分量值进行计算。也就是说,每1 μs添加一个新值,并将包括新值在内的前25个值与40 kHz正弦波进行卷积。25个值的卷积积分非常小,可以每1μs计算一次。因此,这是实时的40 kHz FFT值
\begin{align*}
	X_{40~\text {kHz}} = \left |{\sum {n=1}^{N-1} x{n} \cdot \left ({\cos \left ({-2 \pi k \frac {n}{N}}\right) + i \sin \left ({-2 \pi k \frac {n}{N}}\right)}\right)}\right | \
	{}\tag{9}
\end{align*}
其中X40 kHz是信号中的40 kHz分量的大小,xn是要检查的接收到的电压信号。

本文中使用的参数为
\begin{equation*}
	k = f \times \frac {N}{f_{s}};\quad f = 40~\text {kHz};
	~ f_{s} = 1~\text {MHz};~ N = 25.
\end{equation*}
该值仅表示40 kHz分量的强度,因此仅显示了除去其他噪声的超声波强度,充当噪声滤波器。
\subsubsection{噪声消除}
图6和图7展示了超声波信号与各种噪声之间的差异。超声波信号是从墙壁反射而来的回波信号,距离为5米。环境噪声是在车辆以90 km/h的速度行驶时在道路表面附近收集到的。超声波信号和噪声在不同的实验中收集并进行比较。对于实际应用,超声波传感器系统应能够在车辆高速行驶时拒绝任何噪声。

图6显示了从30 kHz到50 kHz的两个信号的FFT结果。超声波信号和噪声信号都包含在30 kHz到50 kHz范围内。在40 kHz频率附近,噪声比超声波信号强得多。在本文中,从40 kHz FFT值中删除了噪声。

我们每10微秒计算一次40kHz的FFT值,以减少数据量。这在图5中表示为fss,相当于100kHz的子采样或降采样。子采样直接与距离测量分辨率相关,100kHz的子采样具有1.7毫米的测量分辨率。子采样是使用单比特信号的主要优势,而模拟信号分析则需要更高的采样率。子采样频率可以根据所需的距离测量分辨率进行决定,例如每100微秒计算一次,以获得17毫米的距离测量分辨率。
如果40kHz的值大于一个阈值,我们确定接收到了超声波,并将其转换为以下的单比特信号。

1)高:存在超声波信号的状态。

2)低:没有超声波信号。

对于汽车超声波传感器而言,传感器中最大的噪声来自于轮胎和路面的噪声。阈值值越低,由于对低水平信号进行二值化而导致的信息损失越小。因此,阈值被设置为略高于实验获得的噪声值。

图8显示了整体的信号处理过程。图8(a)显示了参考信号,超声波发射器在上升沿时振荡并发射超声脉冲[见图8(b)]。发射的超声脉冲被目标物反射并回波到接收器[见图8(c)],微控制器计算40kHz的FFT值[见图8(d)]。同时进行40kHz值的计算,通过阈值产生单比特信号[见图8(e)]。


单比特信号处理和子采样压缩了数据大小。模拟的32位数字可以被压缩为单比特数字。在本文中,以1MHz进行采样的32位模拟信号被转换为100kHz的单比特信号,从而节省了320倍的存储空间。

在以前的研究[31]中,只比较接收信号与之前接收到的信号之间的间隔,并给出脉冲的唯一性以找到TOF。然而,在本文中,TOF是通过CPPM发射信号与接收信号之间的互相关得到的。两个信号的互相关的最高点的延迟就是TOF。

图8(a)和(e)显示了单比特的发射参考信号和接收信号,它们的互相关显示在图9中。峰值延迟为19.25毫秒,这意味着当接收信号延迟该时间时,接收信号与参考信号最相似。因此,TOF为19.25毫秒,即峰值的延迟,根据公式(1)和(2),距离为3.215米。我们进行了温度校正,激光传感器测得的实际值为3.218米。

图8显示了整体的信号处理过程。图8(a)显示了参考信号,超声波发射器在上升沿时振荡并发射超声脉冲[见图8(b)]。发射的超声脉冲被目标物反射并回波到接收器[见图8(c)],微控制器计算40kHz的FFT值[见图8(d)]。同时进行40kHz值的计算,通过阈值产生单比特信号[见图8(e)]。

由于这两个信号都是二进制的单比特信号,可以缩短处理时间以实现实时互相关计算。与多比特乘法相比,单比特乘法可以显著减少总计算成本,因为互相关涉及大量乘法。单比特乘法可以通过逻辑操作轻松实现。因此,图8(a)和(e)的互相关比图8(b)和(c)创建了更好的实时性能。