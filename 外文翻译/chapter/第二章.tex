\section{外文文献原文}
\subsection{Introduction}
Over time, vehicles, mobile robots, and production lines have been acquiring various sensors. In particular, sensors that measure distance to nearby objects are essential to smart systems. Currently, laser sensors, ultrasonic sensors, stereo cameras, and radio detection and ranging (RADAR) and light detection and ranging systems are widely used as distance sensors. Time-of-flight (TOF) cameras are also being studied to provide 3-D imaging at high frame rates [1]. Among them, ultrasonic sensors are very cheap, and the sensor system is small in size and easy to handle, but due to the physical limitations of ultrasound, the performance, accuracy, and measurement rate are usually lower than with other sensor systems. Accordingly, software research has focused on overcoming the physical limitations.

Most studies on ultrasonic distance sensors have focused on increasing the performance of ultrasonic sensors by adding additional information to the ultrasonic wave. There are many ways to store information in ultrasound signals: for example, frequency modulation, signal coding by pulsewidth modulation, pulse-position modulation (PPM), and phase modulation.

Since the 1980s, many studies have been conducted to prevent collisions. In the field of obstacle avoidance, pioneering studies have been made by Borenstein and Koren [2]–​[5]. After that, efforts were made to improve the accuracy of ultrasonic sensors in the field of robotics considering the multiuser problem. Precise ultrasonic pulse detection methods using various signal processing methods, other than the simple threshold method, have been studied. Barker codes, Golay codes, and Kasami codes were used by several researchers for an ultrasonic local positioning system [6]–​[10]. The cross correlation method was used for pulse detection [11], [12]. Jörg and Berg [13], [15], [16] and Jörg et al. [14] used a pseudorandom code to solve the crosstalk problem. However, these methods for accurate ultrasonic pulse detection are inadequate for automotive applications because of their high computational load, which requires a high-performance microcontroller.

Several researchers introduced binary frequency shift keying modulation to produce emission signals [17]–​[20]. The frequency modulation technique requires wide frequency band piezoelectric transducers, which are usually high priced.

Currently, research on signal coding methods aims to solve the crosstalk problem in the mobile robot multiultrasonic sensor system. In vehicle applications, external crosstalk may occur when the opposite vehicle is using ultrasonic sensors on a narrow one-lane street or when there is an ultrasonic sensor on the side of a vehicle in the other lane on an urban road. Crosstalk is a big problem in practical applications of ultrasonic sensors. A review journal [21] introduced research up to 2005 on crosstalk elimination methods. It is essential to distinguish each ultrasonic sensor to reject crosstalk. When a series of ultrasonic pulse sequences is generated using the chaotic PPM (CPPM), each ultrasonic transducer can be made unique, so it was possible to distinguish each ultrasonic sensor, which means the crosstalk elimination.

Borenstein and Koren [22], [23] introduced the idea of comparison of consecutive readings and comparison with alternating delays called error eliminating rapid ultrasonic firing. Then, as the computer processing capability increased, a cross-correlation method was used to compare the sequences. Fortuna et al. [24], [25] applied CPPM to fire the sonar sensor, and chaotic pulse position-width modulation was used by Yao et al. [26] to obtain better results. To enhance the real-time performance, Meng et al. [27] proposed short optimized PPM sequences. Alonge et al. [28] applied RADAR technology to the ultrasonic sensor system.

Unlike the above-described studies, the focus of this paper was to increase the measurement rate of the ultrasonic sensor system in a multiuser environment. CPPM signals overcome the crosstalk and multiuser problem, and by using the signal coding method, the distance to the object can be measured robustly at a high measurement rate, which is different from the conventional simple TOF method (single pulse echo method). In addition, there are obstacles to overcome for vehicular application. The sensor system must be robust to environmental noise and able to measure moving targets. Basically, it should be available in a multiuser situation and, above all, be cost competitive. In addition, general automotive control applications require a root mean square (rms) error of less than 50 mm and a measurement rate of over 100 Hz.

In a conventional single pulse echo method, the transmitter cannot emit additional ultrasonic pulses while the last pulses are still in flight because each ultrasonic pulse cannot be distinguished. In this paper, the TOF is obtained by comparing ultrasonic pulse sequences rather than comparing single ultrasonic pulses. When calculating the TOF, the time delay was calculated by cross correlation between the transmitted pulse sequence and the received pulse signal. However, cross correlation of a long signal typically requires high computing costs. Therefore, in order to reduce the computational load, single-bit signal processing, introduced by Hirata et al. [29], [30], was used in the cross-correlation process to reduce the computational load and improve the real-time performance. By replacing the multibit products with single-bit logical products in cross correlation, the amount of computation was greatly reduced. In [29] and [30], single-bit signal processing was applied to reduce the computing cost of detecting single ultrasonic pulses. However, in this paper, single-bit signal processing was applied to the cross correlation of CPPM signal sequences.

Section II introduces the hardware system used in the experiment and explains the physical limitations of the classical method. Section III introduces the hardware system used in the experiment and the chaotic system. Section IV deals with signal processing for real-time distance measurement; the fast Fourier transform (FFT) threshold was used for noise cancellation, and subsampling and single-bit signal processing were used to reduce the computational load. The results of the crosstalk rejection simulation are shown in Section V, and the results of the distance measurement experiment showing the measurement rate are given in Section VI.

\subsection{Measurement Method}
Ultrasonic distance sensors use a TOF method. In the simplest way, the controller measures the time from departure to echo arrival. The more advanced methods measure the signal delay by comparing the transmitted signal with the received signal. The signal delay can be measured by cross correlation, which is discussed later in this paper

\begin{equation*} d = C \cdot t_{\text {tof}}/2 \tag{1}\end{equation*}

where d is the target distance, C is the speed of sound in the air, and ttof indicates the flight time of the sound pulse measured through this system. The speed of sound varies with temperature and humidity. The linear model is

\begin{equation*} C = 331.3 + 0.606 \cdot \theta + 0.0124 \cdot H \tag{2}\end{equation*}

where θ is the temperature in degrees Celsius and H is the relative humidity. Temperature changes can cause large changes in the speed of sound, so real-time temperature measurement is basic and is necessary for ultrasonic distance sensing. However, humidity deviations were ignored because they were expected to have a very small impact of less than 0.2 % on the accuracy desired in this paper.

\subsubsection{Single Pulse Echo Method}
The traditional method of ultrasonic distance sensors is called the single pulse echo method. After an ultrasonic transmitter shoots a sound pulse, the controller waits for the echo pulse to measure the flight time ttof . Wasting much time waiting for an echo is inevitable. Since the flight time increases proportionally to the distance, the measurement rate decreases inversely with the distance. Moreover, if the echo disappears due to noise or environmental reasons, more serious problems arise. The sensor cannot shoot the next sound pulses until a timeout.

The maximum measurement rate of the conventional single pulse echo method is as follows:
\begin{equation*} \text {Maximum measurement rate} = \frac {1}{t_{\text {tof}}} = \frac {C}{2 \cdot d}. \tag{3}\end{equation*}

Since the maximum measurement rate is inversely proportional to the distance, it is only 17 Hz for a 10-m target. The main reason for the limitation of the measurement rate of the single pulse method is that it measures with only one sound pulse. If the ultrasonic sensor transmits additional sound pulses while the previous sound pulses are in flight, a wrong TOF will be calculated in the single pulse echo method. Instead of sending the same ultrasound pulse one at a time, it is necessary to modulate the ultrasound signal or the ultrasound pulse sequence to include more information. In a previous study [31], each sound pulse was characterized by its interval to the last pulse. However, this method has the disadvantage of not being robust to noise. This problem can be solved by calculating the TOF by cross correlation of the transmitted and received signals, instead of measuring the round trip time using one ultrasonic pulse.

\subsection{Measurement System}
Fig. 1 shows a diagram of the hardware of the proposed system. Both the transmitter and the receiver used a 40-kHz ultrasonic transducer. A microcontroller was used to run ultrasonic transmitter operation and data acquisition. The microcontroller included a 1-MHz analog-to-digital converter (ADC) that samples a 40-kHz ultrasound wave and a 1.33-GHz dual-core CPU with a field-programmable gate array module. The microcontroller computed the differential equations of Chua’s circuit to generate the chaotic signal for the CPPM and provided a digital pulse sequence to the ultrasonic transmitter. The ultrasound pulses was fired in the predetermined chaotic pulse sequence, the collected ultrasound data were transferred to a computer, and further signal processing was performed on the computer. The analog signal filter is a simple high-pass filter to remove dc components.

A chaotic system was applied to remove crosstalk. The CPPM method was adopted to incorporate chaotic information into a digital pulse sequence for the ultrasound pulse output.

\subsubsection{Chua’s Circuit}

Chua’s circuit [32] is a well-known and the simplest autonomous electronic circuit that generates chaotic signals. Chua’s circuit is shown in Fig. 2(a) and generates a double scroll attractor [33]. It consists of resistors, inductors, capacitors, and a nonlinear resistor called Chua’s diode, whose static characteristic is plotted in 2(b). The state equations of Fig. 2 are
\begin{align*} 
	C_{1} \frac {dv_{1}}{dt}=&G (v_{2} - v_{1}) - f(v_{1}) \\ C_{2} \frac {dv_{2}}{dt}=&G (v_{1} - v_{2}) + i_{L} \\ L \frac {di_{L}}{dt}=&-v_{2} \tag{4}
\end{align*}

where G=1/R . Chua’s diode exhibits a nonlinear function of only one variable, which is different from the Lorenz equation with two nonlinearities, each one being a scalar function of two variables. The nonlinearity function of Chua’s diode is an odd-symmetric piecewise linear function made of three straight line segments
\begin{equation*} f(v_{1})=m_{0}v_{1}+\frac {1}{2}(m_{1}-m_{0})(|v_{1}+B_{p}|-|v_{1}-B_{p}|).\qquad \tag{5}\end{equation*}
Applying the following dimensionless variables:
\begin{align*} x=&\frac {v_{1}}{B_{p}} \quad y=\frac {v_{2}}{B_{p}} ~ z=\frac {i_{L}}{B_{p}G} \\ a=&\frac {m_{1}}{G} \quad b=\frac {m_{0}}{G} \\ \alpha=&\frac {C_{2}}{C_{1}} \quad \beta =\frac {C_{2}}{LG^{2}}\tag{6}\end{align*}
when time scaling is τ=tG/C2 , the dimensionless form is obtained
\begin{align*} \frac {x}{\tau }=&\alpha (y-x-f(x)) \\ \frac {y}{\tau }=&x-y-z \\ \frac {z}{\tau }=&-\beta y \tag{7}\end{align*}
where
\begin{equation*} f(x)=bx+\frac {1}{2}(a-b)(|x+1|-|x-1|). \tag{8}\end{equation*}
The variables, initial values, and time-scaling factor we used are as follows:
\begin{align*} \alpha=&9, \quad \beta =\frac {100}{7} ~a=-\frac {8}{7}~b=-\frac {5}{7}\\ x(0)=&0.09 \quad y(0)=0 ~x(0)=0 ~\frac {G}{C_{2}}=1000.\end{align*}
The graph of state variable x of (7) is drawn in Fig. 3.

\subsubsection{Chaotic Pulse-Position Modulation}
To create a digital signal having a chaotic interpulse interval, the chaotic analog signal goes through a process called PPM, which is called CPPM. The modulation mechanism is shown in Fig. 4. This method was introduced in the ultrasonic distance sensor system by Fortuna et al. [24]. The continuous pulse-position modulator consists of a sample-and-hold circuit and a ramp generator. The chaotic signal is an input to the PPM circuit, and the signal is sampled and held. When the held value minus the increasing ramp signal goes to zero, the circuit generates a digital pulse, resets the ramp signal, and performs sample-and-hold again. The chaotic information of the analog voltage is transferred into the intervals between digital pulses.

\subsubsection{Signal Processing}
Fig. 5 shows the signal processing of the proposed measurement method. The transmitter section of the diagram includes Chua’s circuit and a PPM modulator. The receiver part consists of an ultrasound receiver, a sine wave correlator for noise filtering and threshold for binarization. The transmitter generates ultrasonic pulses according to the CPPM sequence. A transport delay occurs while the ultrasound emitted from the transmitter is reflected on the target and collected by the receiver. The transport delay corresponds to TOF and can be measured by cross correlation of the two signals. Each transmitter and receiver part provides a single-bit signal to the processing unit to calculate the distance. In this paper, the receiver part is configured as a sampled system for further study but can be configured as an analog circuitry as needed.

The FFT was used to determine whether there is an ultrasonic signal in the voltage signal collected by the ultrasonic receiver. The ultrasound used in this paper was 40 kHz (39.4 kHz for exactly). Since another frequency domain of the full FFT is not needed in the incoming voltage signal, if only the 40-kHz component is calculated, the calculation amount is very small and can be calculated quickly. In this paper, the ultrasonic signal ADC speed is 1 MHz, and the period of the 40-kHz ultrasonic wave is 25 μs . Therefore, a new voltage value is obtained every 1 μs , and then, the 40-kHz component value of a 25-μs -long voltage signal is calculated. That is, a new value is added every 1 μs , and the previous 25 values, including the new value, are convoluted with the 40-kHz sine wave. The convolution integral of 25 values is very small and can be calculated every 1μs . Therefore, this is the real-time 40-kHz FFT value
\begin{align*} X_{40~\text {kHz}} = \left |{\sum _{n=1}^{N-1} x_{n} \cdot \left ({\cos \left ({-2 \pi k \frac {n}{N}}\right) + i \sin \left ({-2 \pi k \frac {n}{N}}\right)}\right)}\right | \\ {}\tag{9}\end{align*}
where X40 kHz is the amount of 40 kHz in the signal and xn is the received voltage signal to be examined.

The parameters used in this paper were
\begin{equation*} k = f \times \frac {N}{f_{s}};\quad f = 40~\text {kHz};~ f_{s} = 1~\text {MHz};~ N = 25.\end{equation*}
This value only represents the intensity of the 40-kHz component, so it only displays the intensity of the ultrasonic wave with other noise removed, acting as a noise filter.
\subsubsection{Noise Elimination}
Figs. 6 and 7 show the difference between the ultrasound signals and an ambient noise in various ways. The ultrasonic signal is an echo signal reflected from a wall 5 m away. The ambient noise was collected near the road surface while the vehicle was running at 90 km/h. The ultrasonic signal and the noise were collected in separate experiments and compared. For practical applications, the ultrasonic sensor system should be able to reject any noise while the vehicle is running at high speed.
Fig. 6 shows the FFT result of two signals from 30 to 50 kHz. The ultrasonic signal has a much higher peak value than the noise signal. When the 40-kHz component is analyzed in real time [see Fig. 7(b)], it is easier to distinguish an ultrasonic component from noise than when only the voltage value is analyzed [see Fig. 7(a)]. This allows external noise, such as noise from the road surface, to be filtered.

We calculated the 40-kHz FFT value per 10 μs to reduce the amount of data. This is fss in Fig. 5 and is 100-kHz subsampling or downsampling. The subsampling is directly related to the distance measurement resolution, and the subsampling of 100 kHz has a measurement resolution of 1.7 mm. Subsampling is a main advantage of using single-bit signals, while analog signal analysis requires much high sampling rates. Subsampling frequency can be decided according to the desired distance measurement resolution, such as calculating every 100 μs to get 17-mm distance measurement resolution.
If the value of 40 kHz was larger than a threshold, we determined that the ultrasonic wave was received and converted into the single-bit signal as follows.

1)High: A state in which an ultrasonic signal exists.

2)Low: No ultrasonic signal.

In the case of ultrasonic sensors for automobiles, the largest noise coming into the sensor is the tire-road noise. The lower the threshold value is, the less information loss due to the binarization of the low-level signal can be minimized. Therefore, the threshold was set to a value that was just above the noise obtained from the experiment.

Fig. 8 shows the overall signal processing process. Fig. 8(a) shows the reference signal, and the ultrasonic transmitter oscillates at the rising edge and emits an ultrasonic pulse [see Fig. 8(b)]. The transmitted ultrasonic pulses are reflected back to the target and echoed into the receiver [see Fig. 8(c)], and the microcontroller calculates the 40-kHz FFT value [see Fig. 8(d)]. Simultaneously with the 40-kHz value calculation, a single-bit signal was generated by a threshold [see Fig. 8(e)].

Single-bit signal processing and subsampling compress the data size. Analog 32-bit numbers can be compressed into single-bit number. In this paper, a 32-bit analog signal measured at 1 MHz was converted to a 100-kHz single-bit signal, resulting in 320 times memory savings.

In a previous study [31], only the interval between the received signal and the previously received signal was compared, and the uniqueness of the pulse was given to find the TOF. However, in this paper, the TOF was obtained by the cross correlation between the CPPM transmitted signal and the received signal. The delay, at the highest point of the cross correlation of the two signals, is the TOF.

Fig. 8(a) and (e) shows the single-bitted transmitted reference signal and the received signal, and their cross correlation is shown in Fig. 9. The peak delay is 19.25 ms, which means that the received signal is the most similar to the reference signal when it is delayed by that amount. Therefore, the TOF is 19.25 ms, which is the delay of the peak, and the distance is 3.215 m according to (1) and (2). We computed a temperature correction, and the actual value measured by the laser sensor was 3.218 m.

Since both the signals are binary single-bit signals, the processing time can be shortened to enable real-time cross-correlation calculations. Single-bit multiplication can significantly reduce the total calculation cost compared with multibit multiplication [29], because cross correlation involves a large number of multiplications. Single-bit multiplication can be easily done by logical operations. Therefore, cross correlation of Fig. 8(a) and (e) created a better real-time performance than Fig. 8(b) and (c).
